\documentclass{article}

\usepackage{amsthm,amsmath,amssymb,indentfirst,float}
\usepackage{verbatim}
\usepackage[sort,longnamesfirst]{natbib}
\newcommand{\pcite}[1]{\citeauthor{#1}'s \citeyearpar{#1}}
\newcommand{\ncite}[1]{\citeauthor{#1}, \citeyear{#1}}

\usepackage{geometry}
%\geometry{hmargin=1.025in,vmargin={1.25in,2.5in},nohead,footskip=0.5in} 
%\geometry{hmargin=1.025in,vmargin={1.25in,0.75in},nohead,footskip=0.5in} 
\geometry{hmargin=2.5cm,vmargin={2.5cm,2.5cm},nohead,footskip=0.5in}

\renewcommand{\baselinestretch}{1.25}

\usepackage{amsbsy,amsmath,amsthm,amssymb,graphicx}

\setlength{\baselineskip}{0.3in} \setlength{\parskip}{.05in}


\newcommand{\gbar}{\bar g}
\newcommand{\cvgindist}{\overset{\text{d}}{\longrightarrow}}
\DeclareMathOperator{\PR}{Pr} \DeclareMathOperator{\var}{Var}
\DeclareMathOperator{\cov}{Cov}
\newcommand{\eps}{\epsilon}
\newtheorem{claim}{Claim}

\newcommand{\sX}{{\mathsf X}}
\newcommand{\tQ}{\tilde Q}
\newcommand{\cU}{{\cal U}}
\newcommand{\cX}{{\cal X}}
\newcommand{\tbeta}{\tilde{\beta}}
\newcommand{\tlambda}{\tilde{\lambda}}
\newcommand{\txi}{\tilde{\xi}}



\def\baro{\vskip  .2truecm\hfill \hrule height.5pt \vskip  .2truecm}
\def\barba{\vskip -.1truecm\hfill \hrule height.5pt \vskip .4truecm}

\title{R package glmm}

\author{Christina Knudson}

\begin{document}
\maketitle{}

\section{Bacteria example}
The MASS package contains the command glmmPQL and the bacteria data-set. The manual describes the data set as follows: ``Tests of the presence of the bacteria H. influenzae in children with otitis media in the Northern Territory of Australia.''

The data were fit using glmmPQL, glmm with $m=10^3$, glmm with $m=10^4$, glmm with $m=10^5$. (The data did need to be reformatted, which took only a couple minutes). The parameter estimates are summarized in the following table. More model details can be seen in the output that follows the table.

\begin{tabular}{lccccc}
& Intercept & trtdrug & trtdrug+ & I(week$>2$) TRUE & $\nu$ \\ \hline
glmmPQL & 3.41 & -1.25 & -.75 & -1.61 & 1.99\\
glmm $m=10^3$ & 3.02 & -1.20 & -.89 & -1.44 & .81 \\
glmm $m=10^4$ & 3.65 & -1.36 & -.92 & -1.66 & 2.02 \\
glmm $m=10^5$ & 3.49& -1.65 & -.90 & -1.46 & .90 \\
\end{tabular}

It's safe to say that the bacteria glmm results with a paltry Monte Carlo sample size of $m=10^3$ are not reliable. We conclude this because the estimates change quite a bit when $m=10^4$. The results using $m=10^4$ are very similar to the glmmPQL results.

The glmmPQL results:
\begin{verbatim}
> bac.pql<-glmmPQL(y ~ trt + I(week > 2), random = ~ 1 | ID,
+                 family = binomial, data = bacteria)
> summary(bac.pql)
Linear mixed-effects model fit by maximum likelihood
 Data: bacteria 
  AIC BIC logLik
   NA  NA     NA

Random effects:
 Formula: ~1 | ID
        (Intercept)  Residual
StdDev:    1.410637 0.7800511

Variance function:
 Structure: fixed weights
 Formula: ~invwt 
Fixed effects: y ~ trt + I(week > 2) 
                    Value Std.Error  DF   t-value p-value
(Intercept)      3.412014 0.5185033 169  6.580506  0.0000
trtdrug         -1.247355 0.6440635  47 -1.936696  0.0588
trtdrug+        -0.754327 0.6453978  47 -1.168779  0.2484
I(week > 2)TRUE -1.607257 0.3583379 169 -4.485311  0.0000
 Correlation: 
                (Intr) trtdrg trtdr+
trtdrug         -0.598              
trtdrug+        -0.571  0.460       
I(week > 2)TRUE -0.537  0.047 -0.001

Standardized Within-Group Residuals:
       Min         Q1        Med         Q3        Max 
-5.1985361  0.1572336  0.3513075  0.4949482  1.7448845 

Number of Observations: 220
Number of Groups: 50 
\end{verbatim}

The bacteria glmm results with a  Monte Carlo sample size of $m=10^3$:
\begin{verbatim}
> set.seed(1234)
> bac.glmm1<-glmm(y2~trt+I(week > 2),list(~0+ID), 
family=bernoulli.glmm, data=bacteria, m=10^3, varcomps.names=c("ID"))

> summary(bac.glmm1)

Call:
glmm(fixed = y2 ~ trt + I(week > 2), random = list(~0 + ID),  varcomps.names = c("ID"), 
data = bacteria, family.glmm = bernoulli.glmm,     m = 10^3)

Fixed Effects:
                Estimate Std. Error z value Pr(>|z|)    
(Intercept)       3.0199     0.4667   6.471 9.75e-11 ***
trtdrug          -1.1989     0.4473  -2.680 0.007357 ** 
trtdrug+         -0.8883     0.4742  -1.873 0.061021 .  
I(week > 2)TRUE  -1.4407     0.4258  -3.384 0.000716 ***
---
Signif. codes:  0 ‘***’ 0.001 ‘**’ 0.01 ‘*’ 0.05 ‘.’ 0.1 ‘ ’ 1


Variance Components for Random Effects (P-values are one-tailed):
   Estimate Std. Error z value Pr(>|z|)/2    
ID   0.8175     0.1635   4.998   2.89e-07 ***
---
Signif. codes:  0 ‘***’ 0.001 ‘**’ 0.01 ‘*’ 0.05 ‘.’ 0.1 ‘ ’ 1



\end{verbatim}
The results from fitting the bacteria dataset with glmm and $m=10^4$:

\begin{verbatim}
> set.seed(1234)
> bac.glmm2<-glmm(y2~trt+I(week > 2),list(~0+ID), family=bernoulli.glmm, data=bacteria, m=10^4, varcomps.names=c("ID"))
> summary(bac.glmm2)

Call:
glmm(fixed = y2 ~ trt + I(week > 2), random = list(~0 + ID), 
    varcomps.names = c("ID"), data = bacteria, family.glmm = bernoulli.glmm, 
    m = 10^4)

Fixed Effects:
                Estimate Std. Error z value Pr(>|z|)    
(Intercept)       3.6514     0.6327   5.771  7.9e-09 ***
trtdrug          -1.3645     0.7461  -1.829 0.067427 .  
trtdrug+         -0.9186     0.7619  -1.206 0.227915    
I(week > 2)TRUE  -1.6660     0.4645  -3.587 0.000334 ***
---
Signif. codes:  0 ‘***’ 0.001 ‘**’ 0.01 ‘*’ 0.05 ‘.’ 0.1 ‘ ’ 1


Variance Components for Random Effects (P-values are one-tailed):
   Estimate Std. Error z value Pr(>|z|)/2   
ID   2.0244     0.7515   2.694    0.00353 **
---
Signif. codes:  0 ‘***’ 0.001 ‘**’ 0.01 ‘*’ 0.05 ‘.’ 0.1 ‘ ’ 1

\end{verbatim}

Results  of glmm with $m=10^5$:
\begin{verbatim}
> set.seed(1234)
> bac.glmm3<-glmm(y2~trt+I(week > 2),list(~0+ID), family=bernoulli.glmm, data=bacteria, m=10^5, varcomps.names=c("ID"))
> summary(bac.glmm3)

Call:
glmm(fixed = y2 ~ trt + I(week > 2), random = list(~0 + ID), 
    varcomps.names = c("ID"), data = bacteria, family.glmm = bernoulli.glmm, 
    m = 10^5)

Fixed Effects:
                Estimate Std. Error z value Pr(>|z|)    
(Intercept)       3.4888     0.4861   7.178 7.08e-13 ***
trtdrug          -1.6526     0.5232  -3.159 0.001586 ** 
trtdrug+         -0.8968     0.5160  -1.738 0.082196 .  
I(week > 2)TRUE  -1.4614     0.4334  -3.372 0.000746 ***
---
Signif. codes:  0 ‘***’ 0.001 ‘**’ 0.01 ‘*’ 0.05 ‘.’ 0.1 ‘ ’ 1


Variance Components for Random Effects (P-values are one-tailed):
   Estimate Std. Error z value Pr(>|z|)/2    
ID   0.8999     0.2471   3.641   0.000136 ***
---
Signif. codes:  0 ‘***’ 0.001 ‘**’ 0.01 ‘*’ 0.05 ‘.’ 0.1 ‘ ’ 1


\end{verbatim}

\section{Herd CBPP Example}
The cbpp dataset is located in the lme4 package. The lme4 package describes it thusly: ``Contagious bovine pleuropneumonia (CBPP) is a major disease of cattle in Africa, caused by a
mycoplasma. This dataset describes the serological incidence of CBPP in zebu cattle during a
follow-up survey implemented in 15 commercial herds located in the Boji district of Ethiopia. The
goal of the survey was to study the within-herd spread of CBPP in newly infected herds. Blood
samples were quarterly collected from all animals of these herds to determine their CBPP status.
These data were used to compute the serological incidence of CBPP (new cases occurring during a given time period). Some data are missing (lost to follow-up).''

First, I fit the data using glmer in lme4. Then I fit the data using glmm with $m=10^4$. This took 16.75 minutes on my netbook. The point estimates and the standard errors were very similar between the two  methods of model-fitting. (The data did need to be reformatted, which took only a couple minutes)

First, the results from glmer:
\begin{verbatim}
> summary(gm1)
Generalized linear mixed model fit by maximum likelihood (Laplace
  Approximation) [glmerMod]
 Family: binomial ( logit )
Formula: cbind(incidence, size - incidence) ~ period + (1 | herd)
   Data: cbpp

     AIC      BIC   logLik deviance df.resid 
   194.1    204.2    -92.0    184.1       51 

Scaled residuals: 
    Min      1Q  Median      3Q     Max 
-2.3816 -0.7889 -0.2026  0.5142  2.8791 

Random effects:
 Groups Name        Variance Std.Dev.
 herd   (Intercept) 0.4123   0.6421  
Number of obs: 56, groups: herd, 15

Fixed effects:
            Estimate Std. Error z value Pr(>|z|)    
(Intercept)  -1.3983     0.2312  -6.048 1.47e-09 ***
period2      -0.9919     0.3032  -3.272 0.001068 ** 
period3      -1.1282     0.3228  -3.495 0.000474 ***
period4      -1.5797     0.4220  -3.743 0.000182 ***
---
Signif. codes:  0 ‘***’ 0.001 ‘**’ 0.01 ‘*’ 0.05 ‘.’ 0.1 ‘ ’ 1

Correlation of Fixed Effects:
        (Intr) perid2 perid3
period2 -0.363              
period3 -0.340  0.280       
period4 -0.260  0.213  0.198

\end{verbatim}

Next, the results using glmm with $m=10^4$:
\begin{verbatim}
> summary(herd.glmm1)

Call:
glmm(fixed = Y ~ period, random = list(~0 + herd), varcomps.names = c("herd"), 
    data = herddat, family.glmm = bernoulli.glmm, m = 10^4)

Fixed Effects:
            Estimate Std. Error z value Pr(>|z|)    
(Intercept)  -1.4166     0.2410  -5.879 4.14e-09 ***
period2      -0.9921     0.3078  -3.223 0.001271 ** 
period3      -1.1289     0.3277  -3.445 0.000571 ***
period4      -1.5789     0.4293  -3.678 0.000235 ***
---
Signif. codes:  0 ‘***’ 0.001 ‘**’ 0.01 ‘*’ 0.05 ‘.’ 0.1 ‘ ’ 1


Variance Components for Random Effects (P-values are one-tailed):
     Estimate Std. Error z value Pr(>|z|)/2  
herd   0.4354     0.2488    1.75     0.0401 *
---
Signif. codes:  0 ‘***’ 0.001 ‘**’ 0.01 ‘*’ 0.05 ‘.’ 0.1 ‘ ’ 1

\end{verbatim}


\section{Salamander}



\begin{verbatim}
> set.seed(1234)
> sal.glmm3<-glmm(Mate~Cross,random=list(~0+Female,~0+Male),varcomps.names=c("F","M"),data=salamander,family.glmm=bernoulli.glmm,m=10^4)
> summary(sal.glmm3)

Call:
glmm(fixed = Mate ~ Cross, random = list(~0 + Female, ~0 + Male), 
    varcomps.names = c("F", "M"), data = salamander, family.glmm = bernoulli.glmm, 
    m = 10^4)

Fixed Effects:
            Estimate Std. Error z value Pr(>|z|)    
(Intercept)   1.1123     0.2593   4.290 1.79e-05 ***
CrossRW      -0.5545     0.3558  -1.558    0.119    
CrossWR      -3.1701     0.4029  -7.867 3.62e-15 ***
CrossWW      -0.2122     0.3681  -0.577    0.564    
---
Signif. codes:  0 ‘***’ 0.001 ‘**’ 0.01 ‘*’ 0.05 ‘.’ 0.1 ‘ ’ 1


Variance Components for Random Effects (P-values are one-tailed):
  Estimate Std. Error z value Pr(>|z|)/2    
F   1.2087     0.2534   4.769   9.24e-07 ***
M   0.9485     0.1809   5.244   7.86e-08 ***
---
Signif. codes:  0 ‘***’ 0.001 ‘**’ 0.01 ‘*’ 0.05 ‘.’ 0.1 ‘ ’ 1

\end{verbatim}

Here are the results with $m=10^4$ but I forgot to set the seed.
\begin{verbatim}
> summary(sal.glmm2)

Call:
glmm(fixed = Mate ~ Cross, random = list(~0 + Female, ~0 + Male), 
    varcomps.names = c("F", "M"), data = salamander, family.glmm = bernoulli.glmm, 
    m = 10^4)

Fixed Effects:
            Estimate Std. Error z value Pr(>|z|)    
(Intercept)  0.93879    0.26963   3.482 0.000498 ***
CrossRW     -0.76157    0.37376  -2.038 0.041593 *  
CrossWR     -2.72491    0.40736  -6.689 2.24e-11 ***
CrossWW      0.02256    0.37902   0.060 0.952541    
---
Signif. codes:  0 ‘***’ 0.001 ‘**’ 0.01 ‘*’ 0.05 ‘.’ 0.1 ‘ ’ 1


Variance Components for Random Effects (P-values are one-tailed):
  Estimate Std. Error z value Pr(>|z|)/2    
F   1.5991     0.2965   5.394   3.45e-08 ***
M   0.9908     0.1885   5.255   7.41e-08 ***
---
Signif. codes:  0 ‘***’ 0.001 ‘**’ 0.01 ‘*’ 0.05 ‘.’ 0.1 ‘ ’ 1
\end{verbatim}


\section{Reformatting the Datsets}
\subsection{Bacteria Reformatting}
The issue with the bacteria dataset is the response was y/n rather than 1/0. I created a new response that changed the y to 1 and the n to 0.

\begin{verbatim}
bacteria$y2<-as.integer(bacteria$y)-1
\end{verbatim}
\subsection{CBPP Reformatting}
The cbpp dataset was created for binomial but my package was written for Bernoulli responses. In other words, my package needs a row for each success or failure. I did this in the following way:
\begin{verbatim}
cbpp$nonincidence<-cbpp$size-cbpp$incidence #number of "failures"
herddat<-matrix(data=NA,nrow=842,ncol=3)
colnames(herddat)<-c("Y","period","herd")
rowid<-1
for(i in 1:nrow(cbpp)){
	#make a row for each one of the incidences 
	ntimes<-cbpp[i,2]
	if(ntimes>0){	
		for(j in 1:ntimes){
			herddat[rowid,]<-c(1,cbpp[i,4],cbpp[i,1])
			rowid<-rowid+1
		}
	}
	
	#make a row for each of the nonincidences
	ntimes<-cbpp[i,5]
	if(ntimes>0){	
		for(j in 1:ntimes){
			herddat[rowid,]<-c(0,cbpp[i,4],cbpp[i,1])
			rowid<-rowid+1
		}
	}
}
herddat<-as.data.frame(herddat)
herddat$herd<-as.factor(herddat$herd)
herddat$period<-as.factor(herddat$period)
\end{verbatim}
\end{document}