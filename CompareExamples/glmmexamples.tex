\documentclass{article}

\usepackage{amsthm,amsmath,amssymb,indentfirst,float}
\usepackage{verbatim}
\usepackage[sort,longnamesfirst]{natbib}
\newcommand{\pcite}[1]{\citeauthor{#1}'s \citeyearpar{#1}}
\newcommand{\ncite}[1]{\citeauthor{#1}, \citeyear{#1}}

\usepackage{geometry}
%\geometry{hmargin=1.025in,vmargin={1.25in,2.5in},nohead,footskip=0.5in} 
%\geometry{hmargin=1.025in,vmargin={1.25in,0.75in},nohead,footskip=0.5in} 
\geometry{hmargin=2.5cm,vmargin={2.5cm,2.5cm},nohead,footskip=0.5in}

\renewcommand{\baselinestretch}{1.25}

\usepackage{amsbsy,amsmath,amsthm,amssymb,graphicx}

\setlength{\baselineskip}{0.3in} \setlength{\parskip}{.05in}


\newcommand{\gbar}{\bar g}
\newcommand{\cvgindist}{\overset{\text{d}}{\longrightarrow}}
\DeclareMathOperator{\PR}{Pr} \DeclareMathOperator{\var}{Var}
\DeclareMathOperator{\cov}{Cov}
\newcommand{\eps}{\epsilon}
\newtheorem{claim}{Claim}

\newcommand{\sX}{{\mathsf X}}
\newcommand{\tQ}{\tilde Q}
\newcommand{\cU}{{\cal U}}
\newcommand{\cX}{{\cal X}}
\newcommand{\tbeta}{\tilde{\beta}}
\newcommand{\tlambda}{\tilde{\lambda}}
\newcommand{\txi}{\tilde{\xi}}



\def\baro{\vskip  .2truecm\hfill \hrule height.5pt \vskip  .2truecm}
\def\barba{\vskip -.1truecm\hfill \hrule height.5pt \vskip .4truecm}

\title{R package \texttt{glmm}}

\author{Christina Knudson}

\begin{document}
\maketitle{}

\section{Bacteria example}
The MASS package contains the command \texttt{glmmPQL} and the bacteria data-set. The manual describes the data set as follows: ``Tests of the presence of the bacteria H. influenzae in children with otitis media in the Northern Territory of Australia.''

The data were fit using \texttt{glmmPQL},  \texttt{glmer}  from the \texttt{lme4} package (method implemented was Laplace approximation),  \texttt{glmm} with $m=10^4$,  \texttt{glmm} with $m=10^5$, and \texttt{glmm} with $m=10^6$. (The data did need to be reformatted in a process described at the end of the document). The parameter estimates are summarized in the following table. More model details can be seen in the output that follows the table.

\begin{tabular}{lccccc}
& Intercept & trtdrug & trtdrug+ & I(week$>2$) TRUE & $\nu$ \\ \hline
\texttt{glmmPQL} & 3.41 & -1.25 & -.75 & -1.61 & 1.99\\
\texttt{glmer} & 3.55 & -1.37 & -.78 & -1.60 & 1.54\\
\texttt{glmm} $m=10^4$ & 3.57 & -1.38 & -.80 & -1.62 & 1.66 \\
\texttt{glmm} $m=10^5$ & 3.60& -1.39 & -.79 & -1.63 & 1.73 \\
\texttt{glmm} $m=10^6$ & 3.58& -1.37 & -.79 & -1.63 & 1.71 \\
\end{tabular}



The \texttt{glmmPQL} results:
\begin{verbatim}
> bac.pql<-glmmPQL(y ~ trt + I(week > 2), random = ~ 1 | ID,
+                 family = binomial, data = bacteria)
> summary(bac.pql)
Linear mixed-effects model fit by maximum likelihood
 Data: bacteria 
  AIC BIC logLik
   NA  NA     NA

Random effects:
 Formula: ~1 | ID
        (Intercept)  Residual
StdDev:    1.410637 0.7800511

Variance function:
 Structure: fixed weights
 Formula: ~invwt 
Fixed effects: y ~ trt + I(week > 2) 
                    Value Std.Error  DF   t-value p-value
(Intercept)      3.412014 0.5185033 169  6.580506  0.0000
trtdrug         -1.247355 0.6440635  47 -1.936696  0.0588
trtdrug+        -0.754327 0.6453978  47 -1.168779  0.2484
I(week > 2)TRUE -1.607257 0.3583379 169 -4.485311  0.0000
 Correlation: 
                (Intr) trtdrg trtdr+
trtdrug         -0.598              
trtdrug+        -0.571  0.460       
I(week > 2)TRUE -0.537  0.047 -0.001

Standardized Within-Group Residuals:
       Min         Q1        Med         Q3        Max 
-5.1985361  0.1572336  0.3513075  0.4949482  1.7448845 

Number of Observations: 220
Number of Groups: 50 
\end{verbatim}

\begin{verbatim}
proc.time()-ptm
user  system elapsed 
 16.226   0.044  16.281 
> summary(bac.glmm2)

Call:
glmm(fixed = y2 ~ trt + I(week > 2), random = list(~0 + ID), 
    varcomps.names = c("ID"), data = bacteria, family.glmm = bernoulli.glmm, 
    m = 10^4)

Fixed Effects:
                Estimate Std. Error z value Pr(>|z|)    
(Intercept)       3.5745     0.6371   5.611 2.01e-08 ***
trtdrug          -1.3779     0.6701  -2.056 0.039768 *  
trtdrug+         -0.8014     0.6878  -1.165 0.243950    
I(week > 2)TRUE  -1.6224     0.4674  -3.471 0.000518 ***
---
Signif. codes:  0 ‘***’ 0.001 ‘**’ 0.01 ‘*’ 0.05 ‘.’ 0.1 ‘ ’ 1


Variance Components for Random Effects (P-values are one-tailed):
   Estimate Std. Error z value Pr(>|z|)/2  
ID   1.6635     0.7807   2.131     0.0165 *
---
Signif. codes:  0 ‘***’ 0.001 ‘**’ 0.01 ‘*’ 0.05 ‘.’ 0.1 ‘ ’ 1


\end{verbatim}

Results from  \texttt{glmm}  with $m=10^5$.
\begin{verbatim}
> bac.glmm2<-glmm(y2~trt+I(week > 2),list(~0+ID), family=bernoulli.glmm, data=bacteria, m=10^5, varcomps.names=c("ID"))
> proc.time()-ptm
   user  system elapsed 
175.729   0.150 176.211 
> summary(bac.glmm2)

Call:
glmm(fixed = y2 ~ trt + I(week > 2), random = list(~0 + ID), 
    varcomps.names = c("ID"), data = bacteria, family.glmm = bernoulli.glmm, 
    m = 10^5)

Fixed Effects:
                Estimate Std. Error z value Pr(>|z|)    
(Intercept)       3.6008     0.7666   4.697 2.64e-06 ***
trtdrug          -1.3868     0.7126  -1.946 0.051641 .  
trtdrug+         -0.7940     0.6996  -1.135 0.256348    
I(week > 2)TRUE  -1.6321     0.4917  -3.319 0.000903 ***
---
Signif. codes:  0 ‘***’ 0.001 ‘**’ 0.01 ‘*’ 0.05 ‘.’ 0.1 ‘ ’ 1


Variance Components for Random Effects (P-values are one-tailed):
   Estimate Std. Error z value Pr(>|z|)/2  
ID    1.738      1.317    1.32     0.0935 .
---
Signif. codes:  0 ‘***’ 0.001 ‘**’ 0.01 ‘*’ 0.05 ‘.’ 0.1 ‘ ’ 1

\end{verbatim}

Results from \texttt{glmm} with $m=10^6$.
\begin{verbatim}
 proc.time()-ptm
    user   system  elapsed 
1491.511    1.299 1494.133 
> summary(bac.glmm2)

Call:
glmm(fixed = y2 ~ trt + I(week > 2), random = list(~0 + ID), 
    varcomps.names = c("ID"), data = bacteria, family.glmm = bernoulli.glmm, 
    m = 10^6)

Fixed Effects:
                Estimate Std. Error z value Pr(>|z|)    
(Intercept)       3.5824     0.6952   5.153 2.57e-07 ***
trtdrug          -1.3714     0.6935  -1.978 0.047980 *  
trtdrug+         -0.7929     0.7034  -1.127 0.259644    
I(week > 2)TRUE  -1.6278     0.4801  -3.391 0.000698 ***
---
Signif. codes:  0 ‘***’ 0.001 ‘**’ 0.01 ‘*’ 0.05 ‘.’ 0.1 ‘ ’ 1


Variance Components for Random Effects (P-values are one-tailed):
   Estimate Std. Error z value Pr(>|z|)/2  
ID    1.707      1.050   1.626      0.052 .
---
Signif. codes:  0 ‘***’ 0.001 ‘**’ 0.01 ‘*’ 0.05 ‘.’ 0.1 ‘ ’ 1
\end{verbatim}

Results from \texttt{lme4}.
\begin{verbatim}
 summary(bac.glmer)
Generalized linear mixed model fit by maximum likelihood (Laplace
  Approximation) [glmerMod]
 Family: binomial ( logit )
Formula: y2 ~ trt + I(week > 2) + (1 | ID)
   Data: bacteria

     AIC      BIC   logLik deviance df.resid 
   202.3    219.2    -96.1    192.3      215 

Scaled residuals: 
    Min      1Q  Median      3Q     Max 
-4.5615  0.1359  0.3022  0.4217  1.1276 

Random effects:
 Groups Name        Variance Std.Dev.
 ID     (Intercept) 1.543    1.242   
Number of obs: 220, groups: ID, 50

Fixed effects:
                Estimate Std. Error z value Pr(>|z|)    
(Intercept)       3.5479     0.6958   5.099 3.41e-07 ***
trtdrug          -1.3667     0.6770  -2.019 0.043517 *  
trtdrug+         -0.7826     0.6831  -1.146 0.251926    
I(week > 2)TRUE  -1.5985     0.4759  -3.359 0.000783 ***
---
Signif. codes:  0 ‘***’ 0.001 ‘**’ 0.01 ‘*’ 0.05 ‘.’ 0.1 ‘ ’ 1

Correlation of Fixed Effects:
            (Intr) trtdrg trtdr+
trtdrug     -0.593              
trtdrug+    -0.537  0.487       
I(wk>2)TRUE -0.656  0.126  0.064
> 
\end{verbatim}

\section{Herd CBPP Example}
The cbpp dataset is located in the lme4 package. The lme4 package describes it thusly: ``Contagious bovine pleuropneumonia (CBPP) is a major disease of cattle in Africa, caused by a
mycoplasma. This dataset describes the serological incidence of CBPP in zebu cattle during a
follow-up survey implemented in 15 commercial herds located in the Boji district of Ethiopia. The
goal of the survey was to study the within-herd spread of CBPP in newly infected herds. Blood
samples were quarterly collected from all animals of these herds to determine their CBPP status.
These data were used to compute the serological incidence of CBPP (new cases occurring during a given time period). Some data are missing (lost to follow-up).''

First, I fit the data using glmer in lme4. The model summary says it was fit with Adaptive Hermite Quadrature with 9 quadrature nodes. Then I fit the data using \texttt{glmm}with $m=10^4$. This took 16.75 minutes on my netbook. The point estimates and some of the standard errors are very similar between the two  methods of model-fitting. 

\begin{tabular}{lccccc}
& Intercept & period2 & period3 & period4 & $\nu$ \\ \hline
glmer (numerical integration) & -1.40 & -.99 & -1.13 & -1.58 & .41\\
\texttt{glmm}$m=10^4$ & -1.42 & -.99 & -1.13 & -1.58 & .44 \\
\end{tabular}


Here are the full results from glmer:
\begin{verbatim}
> summary(gm1)
Generalized linear mixed model fit by maximum likelihood (Laplace
  Approximation) [glmerMod]
 Family: binomial ( logit )
Formula: cbind(incidence, size - incidence) ~ period + (1 | herd)
   Data: cbpp

     AIC      BIC   logLik deviance df.resid 
   194.1    204.2    -92.0    184.1       51 

Scaled residuals: 
    Min      1Q  Median      3Q     Max 
-2.3816 -0.7889 -0.2026  0.5142  2.8791 

Random effects:
 Groups Name        Variance Std.Dev.
 herd   (Intercept) 0.4123   0.6421  
Number of obs: 56, groups: herd, 15

Fixed effects:
            Estimate Std. Error z value Pr(>|z|)    
(Intercept)  -1.3983     0.2312  -6.048 1.47e-09 ***
period2      -0.9919     0.3032  -3.272 0.001068 ** 
period3      -1.1282     0.3228  -3.495 0.000474 ***
period4      -1.5797     0.4220  -3.743 0.000182 ***
---
Signif. codes:  0 ‘***’ 0.001 ‘**’ 0.01 ‘*’ 0.05 ‘.’ 0.1 ‘ ’ 1

Correlation of Fixed Effects:
        (Intr) perid2 perid3
period2 -0.363              
period3 -0.340  0.280       
period4 -0.260  0.213  0.198

\end{verbatim}

Next, the results using \texttt{glmm}with $m=10^4$:
\begin{verbatim}
> summary(herd.glmm1)

Call:
glmm(fixed = Y ~ period, random = list(~0 + herd), varcomps.names = c("herd"), 
    data = herddat, family.glmm = bernoulli.glmm, m = 10^4)

Fixed Effects:
            Estimate Std. Error z value Pr(>|z|)    
(Intercept)  -1.4166     0.2410  -5.879 4.14e-09 ***
period2      -0.9921     0.3078  -3.223 0.001271 ** 
period3      -1.1289     0.3277  -3.445 0.000571 ***
period4      -1.5789     0.4293  -3.678 0.000235 ***
---
Signif. codes:  0 ‘***’ 0.001 ‘**’ 0.01 ‘*’ 0.05 ‘.’ 0.1 ‘ ’ 1


Variance Components for Random Effects (P-values are one-tailed):
     Estimate Std. Error z value Pr(>|z|)/2  
herd   0.4354     0.2488    1.75     0.0401 *
---
Signif. codes:  0 ‘***’ 0.001 ‘**’ 0.01 ‘*’ 0.05 ‘.’ 0.1 ‘ ’ 1

\end{verbatim}


\section{Salamander}
The salamander dataset represents success or failure in matings between two types of salamanders denoted by R and W. Four types of crosses are possible: RR, RW, WR, WW. These are the fixed effects. There is a random effect for each female salamander and for each male salamander. Thus, there are two variance components: one for the females and one for the males. The response is binary: whether or not the mating was successful.

I fit the salamander data set using  my \texttt{glmm} package (with $m=10^4$ twice and with $m=10^6$ once)  and compared those results to the MCEM results of Booth and Hobert. They do not look so great, and the $m=10^6$ results took 41 hours to get.

\begin{tabular}{lcccccc}
& Intercept & RW & WR & WW & $\nu_F$ & $\nu_M$ \\ \hline
MCEM (B+H) & 1.03 & .32 & -1.95 & .99 & 1.4 & 1.25 \\
\texttt{glmm} $m=10^5$ &   0.964      -0.645      -2.767      -0.008
\end{tabular}









\section{Murder}
Subjects were asked how many victims of homicide they personally knew. The data set is from Agresti's \textit{Categorical Data Analysis} book. I found similar answers from my glmm package as from \texttt{glmmPQL}.

The results from my package:

\begin{verbatim}
set.seed(1234)
murder.glmm<- glmm(y~race ,random=list(~0+black,~0+white), varcomps.names=c("black","white"), data=murder ,family.glmm=poisson.glmm,m=10^4)

> summary(murder.glmm)

Call:
glmm(fixed = y ~ race, random = list(~0 + black, ~0 + white), 
    varcomps.names = c("black", "white"), data = murder, family.glmm = poisson.glmm, 
    m = 10^4)

Fixed Effects:
            Estimate Std. Error z value Pr(>|z|)    
(Intercept)  0.42003    0.06428   6.534 6.40e-11 ***
raceWhite   -0.33179    0.07021  -4.726 2.29e-06 ***
---
Signif. codes:  0 ‘***’ 0.001 ‘**’ 0.01 ‘*’ 0.05 ‘.’ 0.1 ‘ ’ 1


Variance Components for Random Effects (P-values are one-tailed):
       Estimate Std. Error z value Pr(>|z|)/2
black 6.468e-10  3.452e-09   0.187      0.426
white 2.667e-10  1.596e-09   0.167      0.434
\end{verbatim}

When I make the sample size drastically smaller so that the analysis takes only about a minute, the results are very similar still.
\begin{verbatim}
> summary(murder.glmm)

Call:
glmm(fixed = y ~ race, random = list(~0 + black, ~0 + white), 
    varcomps.names = c("black", "white"), data = murder, family.glmm = poisson.glmm, 
    m = 10^2)

Fixed Effects:
            Estimate Std. Error z value Pr(>|z|)    
(Intercept)  0.42003    0.06428   6.534 6.39e-11 ***
raceWhite   -0.33179    0.07021  -4.726 2.29e-06 ***
---
Signif. codes:  0 ‘***’ 0.001 ‘**’ 0.01 ‘*’ 0.05 ‘.’ 0.1 ‘ ’ 1


Variance Components for Random Effects (P-values are one-tailed):
       Estimate Std. Error z value Pr(>|z|)/2
black 1.190e-09  1.718e-09   0.693      0.244
white 1.070e-08  1.440e-08   0.743      0.229

\end{verbatim}

The murder model resulting from \texttt{glmmPQL}:
\begin{verbatim}
> murder.pql<-glmmPQL(y~race,random=~1|race,data=murder,family=poisson)iteration 1
> summary(murder.pql)
Linear mixed-effects model fit by maximum likelihood
 Data: murder 
  AIC BIC logLik
   NA  NA     NA

Random effects:
 Formula: ~1 | race
         (Intercept)  Residual
StdDev: 9.157794e-06 0.4647694

Variance function:
 Structure: fixed weights
 Formula: ~invwt 
Fixed effects: y ~ race 
                 Value  Std.Error   DF   t-value p-value
(Intercept)  0.4200335 0.02989812 1306  14.04883       0
raceWhite   -0.3317900 0.03265396    0 -10.16079     NaN
 Correlation: 
          (Intr)
raceWhite -0.916

Standardized Within-Group Residuals:
       Min         Q1        Med         Q3        Max 
-0.9104427 -0.1899269 -0.1899269 -0.1899269 12.1624898 

Number of Observations: 1308
Number of Groups: 2 
Warning message:
In pt(-abs(tTable[, "t-value"]), tTable[, "DF"]) : NaNs produced

\end{verbatim}

%\section{Booth and Hobert}
%Here are the results from my glmm package fitting a model to the Booth and Hobert data (with $m=10^4$).
%\begin{verbatim}
%> set.seed(1234)
%> mod.mcml1<-glmm(y~0+x1,list(y~0+z1),varcomps.names=c("z1"),data=BoothHobert,family.glmm=bernoulli.glmm,m=10^4,doPQL=TRUE)
%[1] "PQL estimates have been identified."
%> summary(mod.mcml1)
%
%Call:
%glmm(fixed = y ~ 0 + x1, random = list(y ~ 0 + z1), varcomps.names = c("z1"), 
%    data = BoothHobert, family.glmm = bernoulli.glmm, m = 10^4, 
%    doPQL = TRUE)
%
%Fixed Effects:
%   Estimate Std. Error z value Pr(>|z|)    
%x1    6.225      1.422   4.377  1.2e-05 ***
%---
%Signif. codes:  0 ‘***’ 0.001 ‘**’ 0.01 ‘*’ 0.05 ‘.’ 0.1 ‘ ’ 1
%
%
%Variance Components for Random Effects (P-values are one-tailed):
%   Estimate Std. Error z value Pr(>|z|)/2
%z1    1.907      1.814   1.051      0.147
%
%
%\end{verbatim}
%
%Here are the results from glmmPQL:
%\begin{verbatim}
%> summary(bh)
%Linear mixed-effects model fit by maximum likelihood
% Data: BoothHobert 
%  AIC BIC logLik
%   NA  NA     NA
%
%Random effects:
% Formula: ~1 | z1
%        (Intercept)  Residual
%StdDev:    1.284666 0.8712694
%
%Variance function:
% Structure: fixed weights
% Formula: ~invwt 
%Fixed effects: y ~ 0 + x1 
%      Value Std.Error  DF  t-value p-value
%x1 5.857223  1.008231 140 5.809405       0
%
%Standardized Within-Group Residuals:
%        Min          Q1         Med          Q3         Max 
%-8.55346135  0.07317618  0.17063844  0.35776723  2.15976654 
%
%Number of Observations: 150
%Number of Groups: 10 
%\end{verbatim}


\section{Reformatting the Datsets}
\subsection{Bacteria Reformatting}
The issue with the bacteria dataset is the response was y/n rather than 1/0. I created a new response that changed the y to 1 and the n to 0.

\begin{verbatim}
bacteria$y2<-as.integer(bacteria$y)-1
\end{verbatim}
\subsection{CBPP Reformatting}
The cbpp dataset was created for binomial but my package was written for Bernoulli responses. In other words, my package needs a row for each success or failure. I did this in the following way:
\begin{verbatim}
cbpp$nonincidence<-cbpp$size-cbpp$incidence #number of "failures"
herddat<-matrix(data=NA,nrow=842,ncol=3)
colnames(herddat)<-c("Y","period","herd")
rowid<-1
for(i in 1:nrow(cbpp)){
	#make a row for each one of the incidences 
	ntimes<-cbpp[i,2]
	if(ntimes>0){	
		for(j in 1:ntimes){
			herddat[rowid,]<-c(1,cbpp[i,4],cbpp[i,1])
			rowid<-rowid+1
		}
	}
	
	#make a row for each of the nonincidences
	ntimes<-cbpp[i,5]
	if(ntimes>0){	
		for(j in 1:ntimes){
			herddat[rowid,]<-c(0,cbpp[i,4],cbpp[i,1])
			rowid<-rowid+1
		}
	}
}
herddat<-as.data.frame(herddat)
herddat$herd<-as.factor(herddat$herd)
herddat$period<-as.factor(herddat$period)
\end{verbatim}
\end{document}