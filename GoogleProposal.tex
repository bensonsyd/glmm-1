\documentclass[12pt]{article}

\usepackage{amsthm,amsmath,amssymb,indentfirst,float}
\usepackage{verbatim}
\usepackage[sort,longnamesfirst]{natbib}
\newcommand{\pcite}[1]{\citeauthor{#1}'s \citeyearpar{#1}}
\newcommand{\ncite}[1]{\citeauthor{#1}, \citeyear{#1}}

\usepackage{setspace}
\usepackage[margin=1in]{geometry}

%\geometry{hmargin=2.5cm,vmargin={2.5cm,2.5cm},nohead,footskip=0.5in}
\usepackage{times}

\usepackage{amsbsy,amsmath,amsthm,amssymb,graphicx}

\setlength{\baselineskip}{0.3in} \setlength{\parskip}{.05in}


\newcommand{\gbar}{\bar g}
\newcommand{\cvgindist}{\overset{\text{d}}{\longrightarrow}}
\DeclareMathOperator{\PR}{Pr} \DeclareMathOperator{\var}{Var}
\DeclareMathOperator{\cov}{Cov}
\newcommand{\eps}{\epsilon}
\newtheorem{claim}{Claim}

\newcommand{\sX}{{\mathsf X}}
\newcommand{\tQ}{\tilde Q}
\newcommand{\cU}{{\cal U}}
\newcommand{\cX}{{\cal X}}
\newcommand{\tbeta}{\tilde{\beta}}
\newcommand{\tlambda}{\tilde{\lambda}}
\newcommand{\txi}{\tilde{\xi}}



\def\baro{\vskip  .2truecm\hfill \hrule height.5pt \vskip  .2truecm}
\def\barba{\vskip -.1truecm\hfill \hrule height.5pt \vskip .4truecm}

\title{Generalized Linear Mixed Models via Monte Carlo Likelihood Approximation}

\author{}
\date{}
\doublespacing
\begin{document}
%\maketitle{}
 \centerline{\large \bf Generalized Linear Mixed Models via Monte Carlo Likelihood Approximation} %% Paper title

 \medskip

 \centerline{Christina Knudson\\
christina@umn.edu\\
507-384-2220}
 \smallskip
\noindent{\textbf{Background}}

Generalized linear mixed models (GLMMs) are popular in many fields from ecology to economics. The utility of a GLMM is apparent through a Google search: ``GLMM'' return over 242,000 results.

\noindent{\textbf{Existing R packages}}

The challenge is finding an easy-to-implement and reliable method for fitting and testing two-stage hierarchical models. For very simple problems with just a few random effects, the likelihood can be approximated by numerical integration.  Most models have crossed random effects or too many random effects for numerical integration to handle. Thus, a more commonly used method is penalized quasi-likelihood (PQL), implemented in packages such as lme4 and nlme. However, PQL relies on many unmet assumptions and ``approximations of undetermined accuracy'' and suffers from problematic inferential properties, such as parameter estimates that tend to be too low (\ncite{mccu:sear:2001}). Also, because PQL maximizes a quasi-likelihood rather than the likelihood,  likelihood-based inference (such as testing for whether there is significant variability between doctors in the diabetes example) cannot be performed.  The popularity of PQL despite its inadequacies shows that there is a high demand for tools to fit GLMMs.

\noindent{\textbf{Proposed improvement to methodology}}

As an improvement over PQL, I suggest  Monte Carlo  Likelihood Approximation (MCLA), a computationally intensive method for approximating the likelihood to fit and test GLMMs (\ncite{geyer:thom:1992}).   Because MCLA approximates the entire likelihood, any type of likelihood-based inference can be performed.  Inference such as maximum likelihood or likelihood-ratio testing is standard for many simpler models, but MCLA is the only method that can perform these techniques for GLMMs.  Moreover, MCLA is supported by a rigorous theoretical foundation supplied by \citet{geyer:1994} and  \citet{sung:geyer:2007}. 




Despite MCLA's solid theoretical underpinnings, it is not yet a widely-used technique because there are  few packages available that perform MCLA.  The main problem is that MCLA is  computationally expensive. My proposed package would improve MCLA's efficiency by using an importance sampling distribution that is similar to the true distribution of the random effects.  Random effects are usually assumed to be normally distributed, so all that is left to specify are the mean and variance parameters, which my  package chooses  based on the data at hand.
 



\noindent{\textbf{Progress to date}}

Consider a GLMM with a Poisson or Bernoulli response using the canonical link.  Assume the random effects are independently drawn from a normal distribution with mean 0 and unknown variances. There can be any number of fixed or random effects.  For this setting, I have been working on an R package to approximate and maximize the likelihood via MCLA.   The package is in the testing stage and is nearing completion for the setting described earlier in this paragraph.  This package is part of my doctoral thesis in statistics, which I am earning at the University of Minnesota with Professors Charles Geyer and Galin Jones as my co-advisors.



\noindent{\textbf{Goals and objectives for Google Summer of Code}}

My goals are  (1) to rewrite sections of my package in C to further improve its speed, (2) write functions to perform likelihood ratio tests to compare nested models, (3) write additional functions to fit models with correlated random effects.  These goals are extensions of my current and past work on the topic of using MCLA to fit GLMMs.  Because I have been working on this project and am familiar with the limitations and areas of possible improvement, I am the best individual for this project.  Also, I have access to a wealth of knowledge and experience: one of my advisors is the inventor of MCLA and both of my advisors have submitted and maintained R packages.



\noindent{\textbf{Details and schedule for completion}}

I consider my  goals separately. Completion of one goal does not depend on completion of the other goals, so I could work on the three goals in any order.

(1) Two steps stand out in my package as time-consuming: the step that decides the parameters for the importance sampling distribution and the step that maximizes the likelihood approximation. I learned about using .C to call C from R in a statistical computing course, so I will have those notes to refer to. Because I have written functioning R code that performs these, I will be able to compare the R results to the C results to check my work.  I believe I can accomplish this goal in a month.

(2)  Hypothesis testing for nested models can be split into three cases: the nested models differ in their fixed effects but have the same variance components, the nested models differ by one variance component (and possibly some fixed effects), the nested models differ by two or more variance components (and possibly by some fixed effects). I have worked out the details for calculating the test statistics and p-values for the first two cases and will  be able to write the code for these two in about three weeks.  Coding the last case will take longer because I will need to determine the test statistic and its sampling distribution.  If I do not finish this last piece during the summer, I will finish it in the fall.  

(3) The covariance matrix for the random effects in my currently working code is diagonal, meaning the random effects are independently drawn based on one of possibly many variance components.  I would like to generalize the covariance matrix in order to fit models with time-dependence and location-dependence.  This generality would make my R package more usable and practical.  To fit these types of models, I would like to code two additional variance structures:  first-order autoregressive structure with homogenous variances and exponential decay based on the distance between observations.  I have written my current package with these future changes in mind. Coding and testing with these new covariance structures will take about a month.

I expect to complete this work by August 11.

\noindent{\textbf{Potential significance of research}}

In conclusion, researchers from all fields have created a high demand for a tool to fit and test GLMMs. I would like to make the process of fitting a GLMM as user-friendly and theoretically-grounded as fitting a linear model. My package will allow researchers to find maximum likelihood estimates for GLMMs with various covariance structures and perform likelihood ratio tests to compare the fit of nested models.

\noindent{\textbf{References}}
\vspace{-2.3cm}
\renewcommand{\refname}{}
\bibliographystyle{apalike}
\bibliography{brref}


\end{document}