\documentclass{article}

 \usepackage{url} 
\usepackage{amsthm,amsmath,amssymb,indentfirst,float}
\usepackage{verbatim}
\usepackage[sort,longnamesfirst]{natbib}
\newcommand{\pcite}[1]{\citeauthor{#1}'s \citeyearpar{#1}}
\newcommand{\ncite}[1]{\citeauthor{#1}, \citeyear{#1}}
\DeclareMathOperator{\logit}{logit}
    \DeclareMathOperator{\var}{Var}
   %  \DeclareMathOperator{\det}{det}
     \DeclareMathOperator{\diag}{diag}

\usepackage{geometry}
%\geometry{hmargin=1.025in,vmargin={1.25in,2.5in},nohead,footskip=0.5in} 
%\geometry{hmargin=1.025in,vmargin={1.25in,0.75in},nohead,footskip=0.5in} 
%\geometry{hmargin=2.5cm,vmargin={2.5cm,2.5cm},nohead,footskip=0.5in}

\renewcommand{\baselinestretch}{1.25}

\usepackage{amsbsy,amsmath,amsthm,amssymb,graphicx}

\setlength{\baselineskip}{0.3in} \setlength{\parskip}{.05in}


\newcommand{\cvgindist}{\overset{\text{d}}{\longrightarrow}}
\DeclareMathOperator{\PR}{Pr} 
\DeclareMathOperator{\cov}{Cov}


\newcommand{\sX}{{\mathsf X}}
\newcommand{\tQ}{\tilde Q}
\newcommand{\cU}{{\cal U}}
\newcommand{\cX}{{\cal X}}
\newcommand{\tbeta}{\tilde{\beta}}
\newcommand{\tlambda}{\tilde{\lambda}}
\newcommand{\txi}{\tilde{\xi}}




\title{Design Document for Relevance-Weighting: R Package glmm}

\author{Sydney Benson}

\begin{document}
\maketitle{}

\begin{abstract}
This design document will give an overview of the changes made to the R package \texttt{glmm} with respect to a relevance-weighted likelihood method. We use relevance-weighting to better reflect the real-world occurrence of more or less informative observations.
\end{abstract}

\section{Introduction}
This project is meant to enable the user of the \texttt{glmm} function in the \texttt{glmm} R package to include an optional relevance-weighting scheme. A common assumption of linear models is that each observation in a data set is equally informative and trustworthy; however, in real-world data sets, this is rarely the case. Thus, the optional relevance-weighting scheme will allow the user to place a heavier weight on the more informative and/or trustworthy observations in their data set so that those data points that are less informative affect the model to a lesser degree.

\section{The Process}
First, the function will need to establish whether the user has supplied a proper weighting scheme. Next, the weighting scheme will need to be applied in the \texttt{el.C} function. After defining the weighting vector, the remainder of this section will illustrate how this weighting scheme will be applied. 

\subsection{The Weighting Vector}
This vector, called $\Lambda$, must be a vector with the same length as the response vector and must contain all positive values or zeros. 

\subsection{Weighted Log-Likelihood}
We begin by defining the canonical link function as 

\begin{align}
g(\mu) = \eta = X\beta + Zu
\end{align}

\noindent Additionally, the likelihood is defined as 

\begin{align}
l_m(\theta|y) = \log \left( \dfrac{1}{m} \sum_{k=1}^m \dfrac{f_\theta(u_k,y)}{\tilde{f}(u_k)}
\right)
\end{align}

\noindent and $f_\theta(u_k, y) = f_\theta(y|u_k) \tilde{f_\theta} (u_k)$. We then define $f_\theta(y|u_k)$ as $f_\theta(y|u_k) = \exp \left( \sum_i y_i\eta_i - c(\eta_i) \right)$. Thus, 

\begin{align}
f_\theta (u_k, y) = \exp \left( \sum_i y_i \eta_i - c(\eta_i) \right) \tilde{f_\theta} (u_k)
\end{align}

and

\begin{align}
\Lambda f_\theta (u_k, y) &= \Lambda \exp \left( \sum_i y_i \eta_i - c(\eta_i) \right) \tilde{f_\theta} (u_k) \\
&= \lambda_i \exp \left( \sum_i y_i \eta_i - c(\eta_i) \right) \tilde{f_\theta} (u_k) 
\end{align}

\subsection{Integrating the Weighted Log-Likelihood}
As shown above, the weighting scheme must be accounted for in $f_\theta (u_k, y)$. Thus, the weighting scheme must be applied within the \texttt{el.C} function eventually. However, before we get there, we must follow these steps:

\begin{itemize}
	\item[1.] Write the test for the weighting scheme using the Booth Hobert data set.
	\item[2.] Code a general version of the weighting scheme.
	\item[3.] Re-code the general version of the implementation of the weighting scheme in C.
\end{itemize}

\end{document}
